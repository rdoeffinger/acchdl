\chapter{The ALU core}
\section{Interface of the Accumulator VHDL Entity}
\subsection{Ports of the ALU core}
\newcommand\tabhead[1]{\hline\multicolumn{2}{|c|}{#1}\\ \hline}
\newcommand\tabline[2]{#1 & #2\\ \hline}
\begin{center}
\begin{tabularx}{\textwidth}{|l|X|}
\tabhead     {Basic input signals}
\tabline {clock}     {sensitive to rising edge.}
\tabline {reset}     {set to 1 for an (asynchronous) reset.}
\hline
\tabhead     {Control input signals}
\tabline {sign}      {set to 1 for subtraction instead of addition (flip sign).}
\tabline {op}        {operation to execute. Use one of the op\_ constants.}
\tabline {data\_in}  {data to operate on.}
\tabline {pos}       {block position as signed value needed for some operations.}
\hline
\tabhead     {Output signals}
\tabline {ready}     {
        if 1 indicates that the operation given by the control input signals
        is now being processed and the control input signals should be set to
        the right values for the next operation.
        Use op\_nop if you do not have any next operation to execute (yet).
        The input signals may be changed at any time, not only when ready
        is 1, any value when ready was not 1 will be ignored.}
\tabline {data\_out} {
        result data after any read operation. This will be valid the
        cycle after the ready signal becomes 1 to indicate the start of
        processing for the operation following the read.}
\end{tabularx}
\end{center}

\subsection{Operations of the ALU core}
\label{sec:aluops}
\renewcommand\tabline[2]{#1 & #2\\ \hline}
\begin{center}
\begin{tabularx}{\textwidth}{|l|X|}
\hline
\tabline {Name}  {Function}
\hline
\tabline {op\_nop} {no operation, idle}
\tabline {op\_add} {
    add/subtract (depending on sign signal) a 64 bit block
    (representing a positive integer) at the 32 bit block offset
    specified by the pos signal.}
\tabline {op\_readblock} {
    read 32 bit block specified by the pos signal, reads
    below the actually RAM-backed memory range return 0,
    reads above 0 for positive, X"FFFFFFFF" for negative values.}
\tabline {op\_writeblock} {
    write 32 bit block specified by the pos signal, writes outside
    the implemented range are ignored (ideally they should set overflow/
    underflow flags as appropriate).}
\tabline {op\_readflags} {
    read virtual 32 bit flag register. The upper 16 bits indicate
    which of the lower 16 bits (the actual flags) are valid, which
    allows for compatibility with future extended versions.
    For a list of currently available flags see~\fref{tab:flags}.}
\tabline {op\_writeflags} {
    set the flags marked by a set bit in the upper 16 bits to
    the value in the lower 16 bits.
    This is mostly for allowing to restore the register state from system
    RAM and rarely useful otherwise.
    For a list of available flags see~\fref{tab:flags}.}
\tabline {op\_readoffsets} {
    get exponent offsets for floating point operations.
    Exponent offsets are two 16 bit signed integers.
    Lower 16 bits are for writes (op\_floatadd) higher 16 bits for reads (op\_readfloat).}
\tabline {op\_writeoffsets} {
    set exponent offsets for floating point operations.
    Exponent offsets are two 16 bit signed integers.
    Lower 16 bits are for writes (op\_floatadd) higher 16 bits for reads (op\_readfloat).}
\tabline {op\_readfloat} {
    reads the current register content as a single-precision
    floating-point value. Denormals and $\pm\infty$ are supported, values that should
    be NaN are returned as $\pm\infty$.
    The pos signal is misused to specify the rounding mode, see~\fref{tab:round}.}
\tabline {op\_floatadd} {
    adds/subtracts (depending on sign signal) the given
    single-precision floating-point value. Note that NaN is treated as $\infty$
    currently.}
\end{tabularx}
\end{center}

\begin{table}[ht]
\label{tab:round}
\renewcommand\tabline[2]{#1 & #2\\ \hline}
\begin{center}
\begin{tabular}{|l|l|}
\hline
\tabline {pos signal} {rounding mode}
\hline
\tabline {0}          {to $0$}
\tabline {1}          {away from $0$}
\tabline {2}          {to $-\infty$}
\tabline {3}          {to $+\infty$}
\tabline {4}          {to nearest}
\tabline {other}      {unspecified}
\end{tabular}
\caption{Rounding Modes}
\end{center}
\end{table}

\begin{table}[ht]
\label{tab:flags}
\renewcommand\tabline[2]{#1 & #2\\ \hline}
\begin{center}
\begin{tabular}{|r|l|}
\hline
\tabline {Bit} {Name}
\hline
\tabline {0}   {sign}
\tabline {1}   {overflow}
\tabline {2}   {zero}
\end{tabular}
\caption{ALU flags}
\end{center}
\end{table}

Currently, 3 flags are supported: sign, overflow and zero.\\
The sign and zero flags have their usual meaning, indicating if
the register value is negative or zero. Contrary to normal floating
point values, negative 0 does not exist.\\
Writing to the sign flag only changes the sign while keeping the two's
complement representation the same, so this has probably no practical
use beyond restoring the register state.\\
Clearing the zero flag has no effect, since there is no obvious and non-conflicting
way to give this a meaning. Setting the zero flag clears the register's two's
complement representation. This should always be combined with clearing the sign
flag, otherwise the behaviour (while still well-defined) might be unexpected,
and usually the overflow flag should be cleared as well.\\
The overflow flag indicates any kind of overflow-like situation: adding $\pm\infty$,
adding NaN, adding a value that is too large to fit into the backing RAM (e.g. due
to a large write exponent offset) or an addition that causes an ordinary overflow
of the two's complement representation.\\
By periodically checking and clearing the overflow flag it is possible to
detect summand ranges that contain $\infty$ or NaN values and process these
parts more carefully in an additional step. Note that to make this work well,
adding $\infty$ or NaN has \em no \rm effect except setting the overflow flag.\\
While the overflow flag is set, reads will return $\pm\infty$, after clearing
the flag the result will be the same as if there had been no $\pm\infty$ or NaN
value in the input.\\
In general, any operations that can cause the overflow flag to be set should ensure
that the register value is updated in a way that allows "chaining" of registers
to simplify extension to larger exponent ranges.\\
Chaining means that multiple registers add the same numbers, but use different
write exponent offsets. Like this, these registers together can be used
as one larger register with some additional (possibly software) code to handle
reading the current value.\\
An obvious future extension to these flags would be an "inexact" flag if due to
a negative write exponent offset or addition of double-precision support the
ALU is no longer always exact.

\section{Implementation details}
\subsection{Steps of the op\_floatadd Operation}

\renewcommand\tabline[2]{#1 & \multicolumn{5}{p{0.8\textwidth}|}{#2}\\}
\begin{center}
\begin{tabularx}{\textwidth}{|r|X|r|X|r|X|}
\hline
\tabline {  0} {Handle $\infty$/NaN special cases by setting the overflow bit}
\tabline {  1} {combine sign of value with sign signal}
\tabline {  2} {extract 32 bit block position and shift value from exponent}
\tabline {  3} {extract mantissa (with leading 1 or 0 depending on denormal or not)}
\tabline {  4} {shift mantissa for alignment with 32 bit block boundary}
\tabline {  5} {load first 32 bit block}
\tabline {  6} {add/subtract lower part of mantissa to block value}
\tabline {  7} {store first 32 bit block}
\tabline {  8} {load second 32 bit block}
\tabline {  9} {add/subtract upper part of mantissa and carry to block value}
\tabline { 10} {store second 32 bit block}
\hline\hline
\multicolumn{6}{|p{0.75\textwidth}|}{
    The following steps are only necessary if carry is set,
    but currently they are done always to simplify the implementation}\\
\hline\hline
\tabline { 11} {calculate carry resolution position}
\tabline { 12} {flip bits in-between current position and carry-resolution position}
\hline
13a & load 32 bit block for carry resolution &
\multirow{3}{*}{13b} & \multirow{3}{0.25\textwidth}{flip sign bit as part of carry resolution} &
\multirow{3}{*}{13c} & \multirow{3}{0.25\textwidth}{set overflow bit as part of carry "resolution"}\\
14a & add/subtract carry & & & &\\
15a & store 32 bit block for carry resolution & & & &\\
\hline
\end{tabularx}
\end{center}

These steps are organized into clock cycles as follows:\\
\renewcommand\tabline[3]{#1 & #2 & #3\\}
\begin{center}
\begin{tabular}{|r|l|l|}
\hline
\tabline {Clock} {Steps}                          {State}
\hline
\tabline {1}     {0) - 3)}                        {some final state}
\tabline {2}     {4), 5)}                         {st\_in\_float0}
\tabline {3}     {6), 8), 11), pre-calculate 12)} {st\_add1}
\tabline {4}     {7), 9), 13a)}                   {st\_add2}
\tabline {5}     {10), 12), 14a), 13b), 13c)}     {st\_fixcarry}
\tabline {6}     {15a)}                           {first of next command}
\hline
\end{tabular}
\end{center}
For descriptions of the states, see~\fref{sec:alustate}.\\

Note that processing of the next operation starts with Clock 5, so in
each 4 cycles one floatadd operation can complete.\\
Reducing the number of clock cycles is difficult for several reasons:\\
\begin{itemize}
\item Skipping carry resolution if there is no carry is probably easiest, but
      will increase code complexity since the carry value is only known after
      Clock 4, but to gain anything processing of the next instruction would
      have to start \em at \rm Clock 4.
\item The dual-port BlockRAM resource can not service more read-write requests
      than the current implementation uses, thus any improvements either need
      to use multiple BlockRAMs (which in a previous implementation resulted in
      low clock speeds) or a RAM resource with more than 2 ports.
\item Pipelining will need difficult dependency resolution, since the 32 bit blocks
      that are changed only become known for certain in step 12, though for the
      general case a worst-case guess based on the pre-calculated value for step
      12) may be good enough.
\end{itemize}

\subsection{Steps of the op\_readfloat Operation}

\begin{center}
\begin{tabularx}{\textwidth}{|r|X|}
\hline
 1 & Determine block number of highest set bit (highest bit not equal to sign),
     or the number of the block containing the lowest bit representable by a
     denormal if this is higher.\\
 2 & read block determined in step 1) and the one below.\\
 3 & determine position of first set/unset bit in read block.\\
 4 & calculate exponent value with information from steps 1) and 3).\\
 5 & use value from step 3) to shift the blocks read in 2) so that the highest
     bit unequal to the sign is "leftmost".\\
 6 & determine which is the lowest block that is not all 0.\\
 7 & use results of steps 5) and 6) to check if the value can be represented
     exactly as float, a special case for denormals needs exponent from step 4).\\
 8 & for negative values now invert the value from step 5) (together with rounding
     converts two's complement to absolute value for mantissa).\\
 9 & apply rounding if necessary (needs result from step 7) to avoid bias).\\
10 & adjust exponent for possible carry due to rounding.\\
11 & use sign, exponent from step 10) and mantissa from step 9) to build the float value.
     Needs special-case for denormals and infinities.\\
\hline
\end{tabularx}
\end{center}

These steps are organized into clock cycles as follows:\\
\renewcommand\tabline[3]{#1 & #2 & #3\\}
\begin{center}
\begin{tabularx}{\textwidth}{|r|l|X|}
\hline
\tabline {Clock} {Steps}                          {State}
\hline
\tabline {1}     {previous operation finishes}    {some final state}
\tabline {1a}    {wait if write-back pending}     {st\_out\_float0}
\tabline {2}     {1)}                             {st\_out\_float0}
\tabline {3}     {2), 6)}                         {st\_out\_float1}
\tabline {4}     {2), 3)}                         {st\_out\_float2}
\tabline {5}     {4)}                             {st\_out\_float3}
\tabline {6}     {4), 5), 7), 8)}                 {st\_out\_float4}
\tabline {7}     {9), 10), 11)}                   {st\_out\_float\_normal or st\_out\_float\_denormal or st\_out\_float\_inf}
\hline
\end{tabularx}
\end{center}
For descriptions of the states, see~\fref{sec:alustate}.\\

\subsection{State machine}
\label{sec:alustate}
\begin{figure}[ht]
\begin{center}
\begin{tikzpicture}
  \tikzstyle{every node}=[draw,shape=ellipse]
  \tikzstyle{op}=[midway,draw=none,sloped,anchor=base]
  \node[shape=rectangle, minimum width=11.5cm, minimum height=7cm] (finals) at (-3mm,1cm) {};
  \newcommand{\dist}{2cm}
  \node (ready)              at (0cm, 0mm)     {st\_ready};
  \node (fixcarry)           at (-25:30mm)     {st\_fixcarry};
  \node (out_block1)         at ( -5:36mm)     {st\_out\_block1};
  \node (in_block)           at ( 15:34mm)     {st\_in\_block};
  \node (out_ofs)            at ( 35:30mm)     {st\_out\_ofs};
  \node (in_ofs)             at ( 90:20mm)     {st\_in\_ofs};
  \node (out_status)         at (145:30mm)     {st\_out\_status};
  \node (in_status)          at (165:34mm)     {st\_in\_status};
  \node (out_float_normal)   at (185:36mm)     {st\_out\_float\_normal};
  \node (out_float_denormal) at (205:30mm)     {st\_out\_float\_denormal};
  \node (out_float_inf)      at (270:20mm)     {st\_out\_float\_inf};
  \foreach \src in {fixcarry,out_block1,in_block,out_ofs,in_ofs,out_status,in_status,
                    out_float_normal,out_float_denormal,out_float_inf}{
    \draw[->] (\src) -- (ready)                node[op] {\small $op\_nop$};
  }
% state transitions inside "final" states
  \draw[->] (finals.north east) -- (in_block.east)        node[op] {\small $op\_writeblock$};
  \draw[->] (finals.north east) -- (out_ofs.north)        node[op] {\small $op\_readoffsets$};
  \draw[->] (finals.north east) -- (in_ofs.north)         node[op] {\small $op\_writeoffsets$};
  \draw[->] (finals.north west) -- (out_status.north)     node[op] {\small $op\_readflags$};
  \draw[->] (finals.north west) -- (in_status.west)       node[op] {\small $op\_writeflags$};
% non-"final" states and transitions
  \node (out_block0)         at ( 43mm, 50mm)  {st\_out\_block0};
  \draw[->] (finals.north) -- (out_block0)     node[op] {\small $op\_readblock$};
  \draw[->] (out_block0.east) |- (out_block1);
  \node (add0)               at ( 40mm,-40mm)  {st\_add0};
  \draw[->] (finals) -- (add0)                 node[op] {\small $op\_add$};
  \node (in_float0)          at ( 20mm,-45mm)  {st\_in\_float0};
  \draw[->] (finals) -- (in_float0)            node[op] {\small $op\_floatadd$};
  \node (add1)               at ( 45mm,-50mm)  {st\_add1};
  \draw[->] (add0) -- (add1);
  \draw[->] (in_float0) -- (add1);
  \node (add2)               at ( 50mm,-30mm)  {st\_add2};
  \draw[->] (add1.east) -- (add2);
  \draw[->] (add2.east) |- (fixcarry);
  \node (out_float0)         at (  4mm,-55mm)  {st\_out\_float0};
  \node (out_float1)         at (-50mm,-60mm)  {st\_out\_float1};
  \node (out_float2)         at (-50mm,-50mm)  {st\_out\_float2};
  \node (out_float3)         at (-45mm,-40mm)  {st\_out\_float3};
  \node (out_float4)         at (-40mm,-30mm)  {st\_out\_float4};
  \draw[->] (finals) -- (out_float0)           node[op] {\small $op\_readfloat$};
  \draw[->] (out_float0) -- (out_float1)       node[op] {\small not $write\_enable$};
  \draw[->] (out_float1) -- (out_float2);
  \draw[->] (out_float2) -- (out_float3);
  \draw[->] (out_float3) -- (out_float4);
  \draw[->] (out_float4) -- (out_float_normal.west) node[op,below=-1mm] {\small $-126 \le e \le 127$};
  \draw[->] (out_float4) -- (out_float_denormal.south) node[op] {\small $e < -126$};
  \draw[->] (out_float4) -- (out_float_inf.west) node[op] {\small $e > 127$};
\end{tikzpicture}

\end{center}
\end{figure}

