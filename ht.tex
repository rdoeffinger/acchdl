\chapter{Basic HyperTransport Concepts}

To ease understanding of some of the implementation approaches, I will describe
some of the basics of HyperTransport and some of the details that are not easy
find elsewhere.\\
For more details and precise descriptions, please refer to the HyperTransport
specification~\cite{htspec}.\\
HyperTransport is a high-speed device interconnect, used in particular by AMD
CPUs.\\
A HyperTransport link always connects exactly two devices, which means reduced
management overhead compared to and ordinary bus, while still allowing to
connect many devices by allowing chaining of links, tunnels and bridges.\\
An optional part of the specification also allows for more complex
HyperTransport networks with more advanced routing.\\
On the physical level, HyperTransport uses differential signalling and
independent links for each communication direction.\\
Link speed can be varied independently in each direction as well, allowing
any combination of link frequencies from 200 to 2600 MHz (DDR) and link widths
of 2 bits to 32 bits (requiring 29 to 199 pins respectively~\cite{htarch}).\\
Each link also uses 3 independent virtual channels (i.e. each has their own
buffers so they can not block each other): posted, nonposted and response.\\
Flow control is done using a credit-based system: each virtual channel has an
associated number of credits on the sender side (representing available buffers
on the receiver side), independent for command and data.\\
Sending a command/data reduces the number of credits on the sender side,
the receiver increases them again by sending appropriate nop command packets
(nops do not use credits, thus no deadlocks are possible).\\
For data, each data packet uses one credit (size can vary between 4 and 64
bytes).\\
There is an optional data credit mode where each 4 bytes of data use up one
credit, but this is not available in the implementation used.\\
The implementation used provides 32 credits for each virtual channel, for both
command and data.\\
HyperTransport devices are configured (e.g. base address) by a set of registers
with same layout and behaviour as PCI configuration registers, so it is easily
possible to use a HyperTransport device just like a PCI device, just with a much
faster and lower latency connection.\\
