\chapter{HyperTransport}

\section{Basic HyperTransport Concepts}

To ease understanding of some of the implementation approaches, I will describe
some of the basics of HyperTransport and some of the details that are not easy
find elsewhere.\\
For more details and precise descriptions, please refer to the HyperTransport
specification~\cite{htspec}.\\
HyperTransport is a high-speed device interconnect, used in particular by AMD
CPUs.\\
A HyperTransport link always connects exactly two devices, which means reduced
management overhead compared to and ordinary bus, while still allowing to
connect many devices by allowing chaining of links, tunnels and bridges.\\
An optional part of the specification also allows for more complex
HyperTransport networks with more advanced routing.\\
On the physical level, HyperTransport uses differential signalling and
independent links for each communication direction.\\
Link speed can be varied independently in each direction as well, allowing
any combination of link frequencies from 200 to 2600 MHz (DDR) and link widths
of 2 bits to 32 bits (requiring 29 to 199 pins respectively~\cite{htarch}).\\
Each link also uses 3 independent virtual channels (i.e. each has their own
buffers so they can not block each other): posted, nonposted and response.\\
Flow control is done using a credit-based system: each virtual channel has an
associated number of credits on the sender side (representing available buffers
on the receiver side), independent for command and data.\\
Sending a command/data reduces the number of credits on the sender side,
the receiver increases them again by sending appropriate nop command packets
(nops do not use credits, thus no deadlocks are possible).\\
For data, each data packet uses one credit (size can vary between 4 and 64
bytes).\\
There is an optional data credit mode where each 4 bytes of data use up one
credit, but this is not available in the implementation used.\\
The implementation used provides 32 credits for each virtual channel, for both
command and data.\\
HyperTransport devices are configured (e.g. base address) by a set of registers
with same layout and behaviour as PCI configuration registers, so it is easily
possible to use a HyperTransport device just like a PCI device, just with a much
faster and lower latency connection.\\

\section{The HyperTransport Core}

The HyperTransport implementation used is the "HT Core" by the Computer
Architecture Group of the University of Mannheim, more specifially the
version 0.9 for 16-bit links from September 2007.\\
This implementation handles HyperTransport data credits in a non-trivial
way, and since its documentation does not explain this clearly, it is
explained here.\\
Credits for commands are restored to the sender automatically when they are
read from the fifo, whereas for data (since it can have varying size) an
explicit data\_complete signal must be used.\\
This signal must be set to high for exactly one cycle each time a data packet
has been completely read. In particular, it must not be set if a command
has no associated data, and it must be set for only one cycle even if reading
all associated data takes multiple cycles, and it must be set after or
exactly when the last data of the packet is shifted out of the FIFO.\\
Since correct handling of this signal increases complexity significantly
and wrong handling can result in the CPU hanging, this together with some
additional simplifications is handled separately in the ht\_simplify module,
desribed in the next section.

\section{The HyperTransport Simplification Layer}

The HyperTransport simplification layer (the module called ht\_simplify)
merges the posted and non-posted queue into one queue, handles setting
the data\_complete signals of the HyperTransport core, splits multi-dword
writes into multiple writes of a single dword and inserts nop
commands into the queue as necessary, thus eliminating the need for
and extra "empty" signal, and provides a signal that indicates
if the current command must be replied to via the response queue.\\
This greatly reduces the complexity and frustration of designing a device
that uses HyperTransport by handling the parts that are most error-prone
and easily lead to a stopped CPU.\\
In the current implementation it has also several drawbacks: byte-sized
are not handled correctly, and the transfer rate is limited to one command
and 32 bit data per clock cycle, whereas the HT Core allows for up to one
command and 64 bytes of data per clock cycle per queue.\\

\section{The Memory-Mapping Interface}
