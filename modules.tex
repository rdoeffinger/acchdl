\chapter{Components of the Hardware Design}

\begin{figure}[ht]
\begin{center}
\begin{tikzpicture}
  \footnotesize
  \tikzstyle{every node}=[rectangle,draw]
  \node[minimum height=7cm] (HT-core)      at (0   ,0) {HT-core};
  \node[minimum height=7cm] (HT-simplify)  at (50mm,0) {HT-simplify};
  \node[minimum height=7cm] (HT-interface) at (90mm,0) {HT-interface};
  \node[minimum height=2cm] (acc1) at (130mm,+25mm) {Accu 1};
  \node[minimum height=2cm] (accn) at (130mm,-25mm) {Accu n};
  \tikzstyle{dots}=[dotted,ultra thick]
  \draw[dots,shorten >=5mm,shorten <=5mm] (acc1) -- (accn);

%% draw connections
  \tikzstyle{every node}=[pos=0.5,draw=none,anchor=base]

%% core <-> simplifier connections
  \newcommand{\mysc}{-3mm}
  \newcommand{\myofs}{32mm}
  \foreach \n/\txt in {  0/p\_cmd,   1/p\_cmd\_empty,   3/p\_data,   4/p\_data\_empty,
                         8/np\_cmd,  9/np\_cmd\_empty, 11/np\_data, 12/np\_data\_empty,
                        18/r\_cmd\_full, 21/r\_data\_full}
    \draw[->] ([yshift=\mysc*\n+\myofs] HT-core.east) -- ([yshift=\mysc*\n+\myofs] HT-simplify.west) node {\txt};
  \foreach \n/\txt in {  2/p\_cmd\_get,   5/p\_data\_get,   6/p\_data\_comp,
                        10/np\_cmd\_get, 13/np\_data\_get, 14/np\_data\_comp,
                        16/r\_cmd,       17/r\_cmd\_put,   19/r\_data,        20/r\_data\_put}
    \draw[<-] ([yshift=\mysc*\n+\myofs] HT-core.east) -- ([yshift=\mysc*\n+\myofs] HT-simplify.west) node {\txt};

%% simplifier <-> mmap interface connections
  \renewcommand{\mysc}{-3mm}
  \renewcommand{\myofs}{16mm}
  \foreach \n/\txt in {  0/cmd, 1/data, 2/addr, 3/needs\_reply, 4/final,
                         9/r\_cmd\_full, 12/r\_data\_full}
    \draw[->] ([yshift=\mysc*\n+\myofs] HT-simplify.east) -- ([yshift=\mysc*\n+\myofs] HT-interface.west) node {\txt};
  \foreach \n/\txt in {  5/cmd\_stop,
                         7/r\_cmd,        8/r\_cmd\_put,   10/r\_data,        11/r\_data\_put}
    \draw[<-] ([yshift=\mysc*\n+\myofs] HT-simplify.east) -- ([yshift=\mysc*\n+\myofs] HT-interface.west) node {\txt};

%% mmap interface <-> accumulators connections
  \renewcommand{\mysc}{-3mm}
  \renewcommand{\myofs}{7mm}
  \foreach \n/\txt in {  0/op, 1/pos, 2/sign, 3/data\_in} {
    \draw[->] ([yshift= 25mm+\mysc*\n+\myofs] HT-interface.east) -- ([yshift=\mysc*\n+\myofs] acc1.west) node {\txt};
    \draw[->] ([yshift=-25mm+\mysc*\n+\myofs] HT-interface.east) -- ([yshift=\mysc*\n+\myofs] accn.west) node {\txt};
  };
  \foreach \n/\txt in {  4/data\_out, 5/ready} {
    \draw[<-] ([yshift= 25mm+\mysc*\n+\myofs] HT-interface.east) -- ([yshift=\mysc*\n+\myofs] acc1.west) node {\txt};
    \draw[<-] ([yshift=-25mm+\mysc*\n+\myofs] HT-interface.east) -- ([yshift=\mysc*\n+\myofs] accn.west) node {\txt};
  };
\end{tikzpicture}.

\end{center}
\caption{Modules and basic architecture of the design}
\label{fig:modules}
\end{figure}

The system consists of 4 distinct parts, each implemented as a
separate VHDL entity (except the HyperTransport core which is written in
Verilog), as shown in~\fref{fig:modules}.\\
The connections between these four parts and also connection between the
HyperTransport core and the FPGA pins are described in a toplevel module
called "acctop" which is not depicted since it only contains connections
and no relevant code of its own.\\
Each of these parts are described in detail in the following sections.

\section{The HyperTransport Core}

The HyperTransport implementation used is the "HT Core" by the Computer
Architecture Group of the University of Mannheim, more specifially the
version 0.9 for 16-bit links from September 2007.\\
This implementation handles HyperTransport data credits in a non-trivial
way, and since its documentation does not explain this clearly, it is
explained here.\\
Credits for commands are restored to the sender automatically when they are
read from the fifo, whereas for data (since it can have varying size) an
explicit data\_complete signal must be used.\\
This signal must be set to high for exactly one cycle each time a data packet
has been completely read. In particular, it must not be set if a command
has no associated data, and it must be set for only one cycle even if reading
all associated data takes multiple cycles, and it must be set after or
exactly when the last data of the packet is shifted out of the FIFO.\\
Since correct handling of this signal increases complexity significantly
and wrong handling can result in the CPU hanging, this together with some
additional simplifications is handled separately in the ht\_simplify module,
described in~\fref{sec:htsimplify}.

\section{The HyperTransport Simplification Layer}
\label{sec:htsimplify}

The HyperTransport simplification layer (the module called ht\_simplify)
merges the posted and non-posted queue into one queue, handles setting
the data\_complete signals of the HyperTransport core, splits multi-dword
reads/writes into multiple reads/writes of a single dword and inserts nop
commands into the queue as necessary, thus eliminating the need for
and extra "empty" signal, and provides a signal that indicates
if the current command must be replied to via the response queue.\\
There is also a "final" signal that is necessary to handle correctly
split reads: all responses up to when the "final" signal is set
must be answered in one single response packet.\\
The "final" signal is set for other commands as well, but it should
not be necessary there (it might help implement correct handling
of byte-sized, masked writes though).\\
This greatly reduces the complexity and frustration of designing a device
that uses HyperTransport by handling the parts that are most error-prone
and easily lead to a stopped CPU.\\
In the current implementation it has also several drawbacks: byte-sized
writes are not handled correctly, and the transfer rate is limited to one command
and 32 bits data per clock cycle, whereas the HT Core allows for up to one
command and 64 bits of data per clock cycle per queue.\\

\section{The Memory-Mapping Interface}

The memory-mapping interface translates HyperTransport read-/write-requests into
operations on the ALUs.\\
Each ALU is assigned its own 4KB address space. The 4KB size was chosen to allow
managing the ALUs via the CPUs MMU.
This allows arbitrary mapping of ALU number as seen by the software to hardware
ALU number, thus e.g. two programs each using ALUs 0-4 in their code can be remapped
via MMU for one to use ALUs 0-4 and the other 5-8, so that no expensive saving and
restoring of ALU state is necessary on program switches.\\
This functionality would need a kernel-level driver though, currently there is only
a purely user-level library available for accessing the functionality.\\

\begin{figure}[ht]
\begin{center}
\begin{tikzpicture}
  \footnotesize
  \newcommand{\rectw}{10mm}
  \newcommand{\recth}{5mm}
  \newcommand\drawblock[3]{\draw (#1*\rectw,#2*\recth) rectangle +(\rectw,\recth);\draw (#1*\rectw+\rectw/2,#2*\recth+\recth/2) node {#3};}
  \newcounter{addr}
  \draw (8*\rectw,0*\recth) node[right] {0};
  \drawblock{0}{0}{add}
  \drawblock{1}{0}{add}
  \drawblock{2}{0}{add}
  \drawblock{3}{0}{add$^e$}
  \drawblock{4}{0}{add$^d$}
  \drawblock{5}{0}{add$^c$}
  \drawblock{6}{0}{add$^b$}
  \drawblock{7}{0}{add$^a$}
  \foreach \y/\text in {1/add,2/sub,3/sub}{
    \setcounter{addr}{8*\y}
    \draw (8*\rectw,\y*\recth) node[right] {\arabic{addr}};
    \foreach \x in {0,...,7}{
      \drawblock{\x}{\y}{\text}
    }
  }
  \setcounter{addr}{8*4}
  \draw (8*\rectw,4*\recth) node[right] {\arabic{addr}};

% reserved command part
  \draw[style=dotted] (0*\rectw,4*\recth) rectangle +(8*\rectw,2*\recth);
  \draw (4*\rectw,5*\recth) node {reserved};

  \draw (8*\rectw,6*\recth) node[right] {512};
% status flags/offsets part
  \drawblock{7}{6}{status}
  \drawblock{6}{6}{offsets}
  \setcounter{addr}{512+8}
  \draw (8*\rectw,7*\recth) node[right] {\arabic{addr}};

% reserved status part
  \draw[style=dotted] (0*\rectw,6*\recth) rectangle +(8*\rectw,2*\recth);
  \draw (4*\rectw,7*\recth) node {reserved};

% main register
  \draw (0*\rectw,8*\recth) rectangle +(8*\rectw,4*\recth);
  \draw (4*\rectw,10*\recth) node {sign-extended, two's-complement register value};
  \draw (8*\rectw,12*\recth) node[right] {1024};

\end{tikzpicture}

\label{fig:memlayout}
\caption{Memory layout for memory-mapped access to ALUs}
\begin{tabular}{ll}
$^a$ & read: current ALU value as float rounded to $0$\\
$^b$ & read: current ALU value as float rounded away from $0$\\
$^c$ & read: current ALU value as float rounded to $-\infty$\\
$^d$ & read: current ALU value as float rounded to $+\infty$\\
$^e$ & read: current ALU value as float rounded to nearest\\
\end{tabular}
\end{center}
\end{figure}

The memory layout of a single ALU is described in~\fref{fig:memlayout}.
Address offsets are indicated on the right side in multiples of 4 bytes.\\
The 4 KB area is split in two 2 KB parts: the lower part to execute operations
on the ALU and the upper part for the status.\\
For the command area only areas that add or subtract the written single-precision
floating-point value and the five blocks marked $^a$ to $^e$ that on read
return the current value as a single-precision floating-point value with different
rounding modes are defined.\\
There many consecutive 32-bit blocks that have the same functionality
(add or subtract) since this allows for write-combining and thus better
utilization of HyperTransport bandwidth.\\
The complete ALU status can be saved and restored simply by reading and writing
the upper 2 KB, starting with the lowest address. Note that since byte-sized
writes are not implemented by the HyperTransport interface you should avoid
using normal memory copy functions like memcpy in C and instead use your own
function that always reads and writes 32 bit blocks or multiples thereof.\\
The meaning of the offset and status blocks in the status area are explained
in~\fref{sec:aluops} under op\_readoffset, op\_writeoffset, op\_readflags
and op\_writeflags.\\
Large parts of both areas are reserved for future extensions, like support for
double precisions, multiply-and-accumulate or even more advanced operations.\\
