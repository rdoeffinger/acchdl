\chapter{HyperTransport Debugging}
Some HyperTransport errors cause a "machine check exception" (MCE) which
the Linux kernel will print as a message containing a hexadecimal
number.\\
And example of such a message is:\\
CPU 0: Machine Check Exception: 4 Bank 4 b200000000070f0f\\
This message can be translated into a clear-text string
with the mcelog program, for above message the result is:\\
\begin{verbatim}
> echo "CPU 0: Machine Check Exception: 4 Bank 4 b200000000070f0f" |\
  mcelog --k8 --ascii
HARDWARE ERROR. This is *NOT* a software problem!
Please contact your hardware vendor
CPU 0 4 northbridge   Northbridge ECC error
  ECC syndrome = 0
STATUS 0 MCGSTATUS 4
\end{verbatim}
This is an example that these messages are not always useful, since
an ECC error and a syndrom of 0 is usually contradictory.\\
A typical condition that causes an MCE is when the response
to a read request times out.\\
Unfortunately by far not all protocol errors cause an MCE. In particular,
answering a read request with "target done" (without data) instead of
a read request causes none, but results in "random" data as a result
of the read.
Note that it is usually not truly random but usually randomly one out of
about 4 different values with a strong bias towards one of those values
--- possibly whatever happened to be in the corresponding cache line or
register before.\\
There is also no error in case a read request for multiple
dwords~\footnote{dword means 32 bits in this context}
is only responded with one single dword.
This actually does not seem to cause any problem at all.
Unfortunately, the HyperTransport specification does
not say anything explicitly about this case, so it remains unclear if
this works just out of luck and with AMD's Opteron HyperTransport
implementation or if all HyperTransport devices must support this.\\
Another type of error causes the CPU to hang without any kind of output
or other error indication.
The cause of these remains unclear, but so far these could all be solved
by changing various synthesis options, usually decreasing optimization
levels.
Rewriting the code to use more processes, preferably changing only
very few signals in each process and avoiding variables also seems to
help.
The precise cause of this behaviour has not been found so far,
even though the dependency on the optimization level hints at a problem
with the synthesis tools, possibly also related to mixing VHDL and
Verilog code.
