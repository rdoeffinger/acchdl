\newcommand*{\fancyreflstlabelprefix}{lst}
\frefformat{vario}{\fancyreflstlabelprefix}{listing\fancyrefdefaultspacing#1#3}
\Frefformat{vario}{\fancyreflstlabelprefix}{Listing\fancyrefdefaultspacing#1#3}
\lstdefinelanguage[x86att]{Assembler}{
  morekeywords={
    add, addl, addq, call, clflush, enter, jmp, leave,
    mov, movaps, movl, movslq, movss, movq,
    pop, popl, push, pushl, ret, sal, sall, sub, subl, xor, xorl
  },
  morecomment=[l];,
  morecomment=[l]\#,
  morestring=[b]",
  morestring=[b]',
  keywordsprefix=\%
}

\lstset{basicstyle=\ttfamily}
\lstset{breaklines=true}
\lstset{captionpos=b,columns=[c]flexible}
\lstset{language=C}

\chapter{The Software Interface}
For accessing the hardware a simple C library interface is provided.
\Fref{lst:efacexample} is a small example program showing the basic
use of this library, derived from one of the test programs for the design.
\lstinputlisting[caption=example code for libefac use,label=lst:efacexample]{examplesum.c}
\section{Available Functions}
\subsection{int efac\_init(void)}
This function initializes the device and clears all registers. Since there is
no kernel-driver available yet, this uses mmap and libpci to directly map the
device into virtual memory and thus needs root permission and the device
can only be used by one process and thread.\\
A return value of 0 means initialization has failed.
\subsection{void efac\_clear(int reg)}
Clears the ALU with number $reg$ to 0, also resetting all flags.
\subsection{void efac\_add(int reg, float val)}
Adds a single-precision floating-point value $val$ to the value
of ALU number $reg$.
\subsection{void efac\_sub(int reg, float val)}
Behaves just as efac\_add just with inverted sign.
\subsection{void efac\_add4(int reg, float val1, float val2, float val3, float val4)}
Same behaviour as calling efac\_add four times, but may be faster.
\subsection{void efac\_sub4(int reg, float val1, float val2, float val3, float val4)}
As efac\_add4 with inverted signs.
\subsection{float efac\_read(int reg)}
Reads the current value of ALU number $reg$ as a single-precision floating-point value.
The rounding mode used in unspecified.
\subsection{float efac\_read\_round\_zero(int reg)}
As efac\_read but the value is rounded towards $0$ if rounding is necessary.
\subsection{float efac\_read\_round\_inf(int reg)}
As efac\_read but the value is rounded away from $0$ if rounding is necessary.
\subsection{float efac\_read\_round\_ninf(int reg)}
As efac\_read but the value is rounded towards $-\infty$ if rounding is necessary.
\subsection{float efac\_read\_round\_pinf(int reg)}
As efac\_read but the value is rounded towards $+\infty$ if rounding is necessary.
\subsection{float efac\_read\_round\_nearest(int reg)}
As efac\_read but the value is rounded towards the nearest single-precision
floating-point number.
To avoid a bias, a number \em exactly \rm in-between two single-precision
floating-point numbers is rounded towards the one with the lowest bit
cleared ("even").
\subsection{int efac\_is\_negative(int reg)}
Returns $1$ if the current value of ALU number $reg$ is $< 0$, otherwise $0$.
\subsection{int efac\_is\_overflow(int reg)}
Returns $1$ if ALU number $reg$ is in overflow state, otherwise $0$.
\subsection{int efac\_is\_zero(int reg)}
Returns $1$ if ALU number $reg$ is $= 0$, otherwise $0$.
\subsection{void efac\_clear\_overflow(int reg)}
Clears the overflow state of ALU number $reg$.
\subsection{void efac\_set\_offsets(int reg, int16\_t read\_offset, int16\_t write\_offset)}
\subsection{void efac\_get\_offsets(int reg, int16\_t *read\_offset, int16\_t *write\_offset)}
\subsection{void efac\_save(int reg, uint32\_t buf[512])}
\subsection{void efac\_restore(int reg, const uint32\_t buf[512])}
\section{Optimization Tricks}
This section will explain some of the very specific tricks used in
order to generate faster and smaller code to access the device.
They are not very useful yet, since the speed of the device itself
is the major limit and no software optimizations help (though smaller
code always means less cache pressure).
Still these details may be interesting to some and should help
understand the code.\\
The following explanations assume at least basic knowledge of x86
assembler and how application linking and operating systems work.\\
For a reference of x86 assembler instructions refer to e.g. AMD's
Programmer's Manuals~\cite{amdinstr}.\\
To demonstrate these various tricks, the code in~\fref{lst:demo}
is compiled in several different ways.\\
All tests were done with "gcc (GCC) 4.1.2 (Ubuntu 4.1.2-0ubuntu4)",
and unless indicate otherwise were build for the x86\_64 architecture.\\
In these examples, the device address range is assumed to be marked
uncacheable which might not be the best option for optimal speed but
simplifies the code for demonstration --- the difference is just a
missing instruction like clflush or mfence.\\


\begin{lstlisting}[float=ht,caption=compilation demonstration code,label=lst:demo]
#include "libefac.h"
void test(void) {
  efac_add4(0, 1.0, 1.0, 1.0, 1.0);
}
\end{lstlisting}

\lstset{language={[x86att]Assembler}}

The C library during efac\_init maps the physical address space of the
device into the application's address space. This is done via mmap,
and usually the choice of virtual address is left to mmap and the
resulting address stored in a variable.\\
This leads to assembler code as in~\fref{lst:regptr}. efac\_regs
here is a pointer variable (declared as "uint32\_t *efac\_regs").
The contents of this pointer variable as a first step must be loaded
into a register.\\
If now efac\_regs is instead declared as an array
("uint32\_t efac\_regs[REGISTER\_SIZE]") and the device simply mapped
"over" it, the address is already known at link time and can be hard-code,
as~\fref{lst:inline64} shows. Note that this trick may fail if compiling
a dynamic library (more precisely if generating position-independent code).\\

\begin{lstlisting}[float=ht,caption={x86\_64 using register pointer (gcc -S -m64 -O3)},label={lst:regptr}]
test:
        movq    efac_regs(%rip), %rdx
        movl    $0x3f800000, %eax
        movl    %eax, 64(%rdx)
        movl    %eax, 68(%rdx)
        movl    %eax, 72(%rdx)
        movl    %eax, 76(%rdx)
        ret
\end{lstlisting}

\begin{lstlisting}[float=ht,caption={x86\_64 inlined (gcc -S -m64 -O3)},label={lst:inline64}]
test:
        movl    $0x3f800000, %eax
        movl    %eax, efac_regs+64(%rip)
        movl    %eax, efac_regs+68(%rip)
        movl    %eax, efac_regs+72(%rip)
        movl    %eax, efac_regs+76(%rip)
        ret
\end{lstlisting}

Another optimization is that except for initialization most of the code is
in the header file, so that the functions can be inlined. The benefit is
not only that the function call overhead is saved, but also expressions
based on constant function arguments can be calculated at compile-time
instead of runtime.\\
In \fref{lst:noinline64} the address must be calculated at runtime,
whereas due to inlining the can be calculated at compile-time in
\fref{lst:inline64}. Of course this pre-calculation is only possible
if the register number is a constant, but this should be the much
more common case --- on most CPU architectures the register number
is coded into the instruction at compile-time as well.\\
Note that the effect of inlining is even more pronounced when compiling
for 32 bit x86 architecture, as a comparison between \fref{lst:inline32}
and \fref{lst:noinline32} shows. This is because the 32 bit ABI passes
all arguments via the stack, which means the function needs a stack
frame and the data must be loaded from stack into a register before
it can be written again to the device.\\


\begin{lstlisting}[float=ht,caption={x86\_64 not inlined (gcc -S -m64 -O3 -fno-inline)},label={lst:noinline64}]
efac_add4:
        sall    $12, %edi
        movslq  %edi,%rdi
        addq    $efac_regs, %rdi
        movss   %xmm0, 64(%rdi)
        movss   %xmm1, 68(%rdi)
        movss   %xmm2, 72(%rdi)
        movss   %xmm3, 76(%rdi)
        ret

test:
        movss   .LC0(%rip), %xmm3
        xorl    %edi, %edi
        movaps  %xmm3, %xmm2
        movaps  %xmm3, %xmm1
        movaps  %xmm3, %xmm0
        jmp     efac_add4
\end{lstlisting}

\begin{lstlisting}[float=ht,caption={x86 inlined (gcc -S -m32 -O3)},label={lst:inline32}]
test:
        movl    $0x3f800000, %eax
        pushl   %ebp
        movl    %esp, %ebp
        movl    %eax, efac_regs+64
        movl    %eax, efac_regs+68
        movl    %eax, efac_regs+72
        movl    %eax, efac_regs+76
        popl    %ebp
        ret
\end{lstlisting}

\begin{lstlisting}[float=ht,caption={x86 not inlined (gcc -S -m32 -O3 -fno-inline)},label={lst:noinline32}]
efac_add4:
        pushl   %ebp
        movl    %esp, %ebp
        movl    8(%ebp), %edx
        sall    $12, %eax
        addl    $efac_regs, %eax
        movl    %edx, 64(%eax)
        movl    12(%ebp), %edx
        movl    %edx, 68(%eax)
        movl    16(%ebp), %edx
        movl    %edx, 72(%eax)
        movl    20(%ebp), %edx
        movl    %edx, 76(%eax)
        popl    %ebp
        ret

test:
        pushl   %ebp
        movl    $0x3f800000, %eax
        movl    %esp, %ebp
        subl    $16, %esp
        movl    %eax, 12(%esp)
        movl    %eax, 8(%esp)
        movl    %eax, 4(%esp)
        movl    %eax, (%esp)
        xorl    %eax, %eax
        call    efac_add4
        leave
        ret
\end{lstlisting}
